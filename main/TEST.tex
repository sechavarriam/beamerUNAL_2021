%\documentclass[aspectratio=169]{beamer} %Aspect ratio 16:9 (160mm:90mm), same as PowerPoint panoramic
\documentclass[]{beamer} %Aspect ratio 4:3

%\usepackage[utf8]{inputenc}
%\usepackage[T1]{fontenc}

\title{Titulo Plantilla de prueba UNAL}
\subtitle{Subtitulo plantilla}

\date[ISPN ’80]{\today}

\author[Sebasti\'an Echavarr\'ia ]{Sebasti\'an Echavarr\'ia\\ \texttt{sechavarriam@unal.edu.co}}

\institute[]{Facultad de Minas - Departamento de Ingeniería Civil - Sede Medellín}

\usetheme{UN}

\begin{document}
	
	\begin{frame}\end{frame} %Frame auxiliar con escudo logo y eslógan
	
	
	\begin{frame} % Pagina titulo
		\titlepage
	\end{frame}

\begin{frame} 
	\tableofcontents[pausesections]
\end{frame}


\section{Prueba de fuentes regulares}

\begin{frame}{Familias de texto normales Ancizar}
	\begin{itemize}
	\item  Texto normal\\
	\item  TEXTO NORMAL MAYÚSCULAS\\
	\item  \textbf{Texto en negrilla}\\
	\item  \textbf{TEXTO NEGRILLA MAYÚSCULAS}\\
	\item  \textit{Texto en cursiva}\\
	\item  \textit{TEXTO EN CURSIVA MAYÚSCULAS}\\
	\item  \textsc{Texto en versalitas}\\
	\item  \textsc{TEXTO EN VERSALITAS MAYÚSCULAS}
	\end{itemize}
	
\end{frame}

\section{Prueba de nuevas familias de fuente Ancizar}
\begin{frame}{Nuevas Familias de Texto}
	\framesubtitle{Sans y Serif FUENTE SEPARADA}
	
	
	{\SansBlackItalic     SansBlackItalic    } \\
	{\SansBlackItalic     SansBlackItalic    } \\
	{\SansBlack           SansBlack          } \\
	{\SansExtraboldItalic SansExtraboldItalic} \\
	{\SansExtrabold       SansExtrabold      } \\
	{\SansLightItalic     SansLightItalic    } \\
	{\SansLight           SansLight          } \\~\\
 	
 	{\SerifExtraboldItalic SerifExtraboldItalic} \\
 	{\SerifExtrabold       SerifExtrabold      } \\
 	{\SerifLightItalic     SerifLightItalic    } \\
 	{\SerifLight           SerifLight          } \\
 	
	
\end{frame}





\begin{frame}
	\frametitle{Titulo}
	\framesubtitle{Subtitulo} 
  zxzxczxczxczxfad
  Texto de prueba
  \begin{equation}
  	\int_{\Omega}^{\infty}f(x)d\mu
  \end{equation}

  \begin{multline}
  	\label{LeastSquares}
  	\dot{\mathbf{\phi}}^h=\min_{\dot{\mathbf{\phi}}\in\mathbb{R}^{n_{I}}} \left\lbrace \left\|\mathbf{r}(w,\mathbf{b},\dot{\mathbf{\phi}}) \right\|_2  \right\rbrace\\
  	=\min_{\dot{\mathbf{\phi}}\in\mathbb{R}^{n_{I}}} \left\lbrace \sum_{i=1}^{n_s} \sqrt{w_i\mathbf{r}_i^2}  \right\rbrace =\min_{\dot{\mathbf{\phi}}\in\mathbb{R}^{n_{I}}} \left\lbrace \sum_{i=1}^{n_s} {w_i\mathbf{r}_i^2}  \right\rbrace.
  \end{multline}

  
\end{frame}
	
\begin{frame}
	\frametitle{There Is No Largest Prime Number}
	\framesubtitle{The proof uses \textit{reductio ad absurdum}.}
	\begin{theorem}
		There is no largest prime number.
	\end{theorem}
	\begin{proof}
		\begin{enumerate}
			\item<1-> Suppose $p$ were the largest prime number.
			\item<2-> Let $q$ be the product of the first $p$ numbers.
			\item<3-> Then $q + 1$ is not divisible by any of them.
			\item<1-> But $q + 1$ is greater than $1$, thus divisible by some prime
			number not in the first $p$ numbers.\qedhere
		\end{enumerate}
	\end{proof}
	\uncover<4->{The proof used \textit{reductio ad absurdum}.}
\end{frame}	
	
	
\end{document}